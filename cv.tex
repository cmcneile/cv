\documentclass[12pt]{article}

%% http://ftp.uni-erlangen.de/mirrors/CTAN/macros/latex/contrib/multibib/multibib.pdf
\usepackage{multibib}
%%\usepackage{multibbl}
\usepackage{epsf}
\usepackage{graphicx}
\usepackage{url}

\newcommand{\amu}{\mbox{$a_\mu^{\mathrm{HVP, LO}}$}}

\usepackage[section]{placeins}

\newcites{ltex}{Journal articles}

\setlength{\oddsidemargin}{-0.25in}
\setlength{\evensidemargin}{-0.25in}
\setlength{\textwidth}{7.0in}
\setlength{\topmargin}{-0.5in}
\setlength{\textheight}{9.0in}

\begin{document}

\title{Curriculum Vitae}

\author{Craig McNeile}

\maketitle

\section{General details.}
\begin{verbatim}
Address :     School of Engineering, Computing and Mathematics,
              Faculty of Science and Engineering,
              University of Plymouth,
              Drake Circus,
              Plymouth,
              PL4 8AA,
              UK             
Email :  craig.mcneile@plymouth.ac.uk
Phone :  (01752) 586332
Date of birth :  18 March 1967
Country of Birth : England
Home page: http://sites.google.com/site/mcneilephysics/
\end{verbatim}

%%\section{Education and work history.}
%

\begin{table}[tb]
\centering
\begin{tabular}{|c|c|c|} \hline
Dates & University  &  Position \\  \hline
2013 -  & University of Plymouth   & Lecturer \\
2009 - 2013 & University of Wuppertal & Postdoc \\
2006 - 2009 & University of Glasgow & SUPA advanced fellow \\
2004 - 2006 & University of Liverpool & Temporay lecturer \\
2002 - 2004 & University of Liverpool & UKQCD software manager, PI Alan Irving \\
2001 - 2002 & University of Liverpool & Physicist programmer, PI Alan Irving \\
1998 - 2001 & University of Liverpool & Postdoc, PI Chris Michael \\
1995 - 1998 & University of Utah & Postdoc, PI Carleton DeTar \\
1992 - 1995 & University of Kentucky & Postdoc, PI Terry Draper \\
1989 - 1992 & University of Edinburgh & Ph.D (supervisors Dr. Bowler and Dr. Pendleton) \\ 
1986 - 1989 & Imperial College & BSc Undergraduate in Physics (first class degree) \\
\hline
\end{tabular}
\label{tb:life}
\caption{Education and work history}
\end{table}

\FloatBarrier

%%My education and work history is in table~\ref{tb:life}.
%
\section{Teaching experience}

\subsection{My experience in lecturing}


\begin{description}

 \item[University of Plymouth, 
Software Development and Databases,
2 years]\hfill \\ This is a 20 credit module taught in
MSc program Data Science and Business Analytics. This is a conversion
MSc, so many students have no prior experience with programming.

  \item[University of Plymouth, Operational Research and Monte Carlo Methods,
9 years] \hfill \\
This course is taught to  30 to 70 students over a semester. The course is fully
assessed via coursework. I started teaching the students with the statistical language R, switched 
to Matlab, and finally I now teach using python.

\item[University of Plymouth, Project, 3 years] I am the module leader
for the final year projects in Mathematical Sciences. I have
supervised three final year undergraduate projects.


  \item[University of Plymouth, Foundation Year Physics,
6 years] \hfill \\
I taught A-level physics to 120 Engineering students
on the foundation year. I also supervised the experimental 
laboratory sessions.

\item[University of Plymouth, Professional Experience in Mathematics
  Education / Education Project ,
1 years] \hfill \\
I marked the essays and contributed to the discussion sessions.

  \item[University of Plymouth, The Quantum Universe,
3 years] \hfill \\
This was an interdisciplinary 20 credit module, which was taught
immersively over a month to first year undergraduate students.. 
I presented material in the planetarium on campus and some lectures on
particle physics and medical applications of physics.

  \item[University of Plymouth, Mathematical Programming,
1 year] \hfill \\
I developed this module and wrote the module record. Students
were taught python and Vpython. Juypter notebooks were used as well
as an online system.

  \item[University of Plymouth, Numerical and Computational Methods,
1 year] \hfill \\
I lectured basic numerical methods,
such as Simpson's rule,
 to 70 students. I also supervised
students in the computer lab, where they worked with Maple
and Matlab.

  \item[University of Plymouth, Engineering Statistics,
2 years] \hfill \\
I taught statistics to 120 final year mechanical engineering 
students. The course was assessed via course-work and a class 
test.

  \item[University of Liverpool, Special and General
Relativity,
2 years] \hfill \\
I taught a course on special and general
relativity to third year undergraduate mathematics students. 
There were 36 lectures and 12 class tutorials.

  \item[University of Liverpool, Mathematical Methods, 2 years] \hfill \\
I taught a mathematical methods course
 to 70 first year undergraduate science students.  The course
content was ``advanced calculus'' and there were 36 lectures 
with 12 class tutorials.


\item[University of Utah, Computational Physics, 2 years] \hfill \\
The course was partly an introduction to numerical
analysis, as well as an introduction to the use of the tools on a UNIX
system to solve scientific problems.  I taught the following topics:
interpolation, numerical solution of ODEs, finding roots of
equations, and numerical linear algebra.  The students wrote simple
programs in C or Fortran. When I taught the course in 1998, the
students were introduced to the C++ programming language. The maple
computer algebra package was used for about half the course.


\end{description}

\subsection{My experience in teaching small groups of undergraduates}


\begin{description}
  \item[University of Wuppertal, Computational Physics, 1 year] \hfill \\
I wrote
the homework problems for an 
undergraduate course on computational
physics. Matlab was used for topics such as conjugate gradient
and the Ising model.

  \item[University of Glasgow, Physics 2X/2Y, 3 years] \hfill \\
I was a course tutor for the Physics 2X/2Y course
for the second year undergraduate students
in the Physics Department. I met with 4 students for 1 hour per week,
and I marked their homework.
The following topics were covered:
physics of waves, dynamics, physics of solids, thermal physics,
electricity and magnetism, nuclear and particle physics, physics of
optics, and mathematical techniques.

\item[University of Liverpool, first year linear algebra
 (M103), 2 years] \hfill \\
I met with a small group of students, discussed homework,
and marked their work.

\item[University of Liverpool, first year calculus (M101), 2 years] \hfill \\
I met with a small group of students, discussed calculations,
and marked their homework.



\item[University of Liverpool, Introduction to Java Programming
 (C101), 3 years] \hfill \\
I worked with the students for 1 hour in the computer lab.
I graded their computer programs and documentation.



\item[University of Liverpool, Essay projects, 2 years] \hfill \\
I supervised a 3rd and 4th year Mathematical Physics project.

\end{description}

\subsection{Teaching MSc students}

At the University of Plymouth, the MSc with the largest
number of students in the Faculty of Science and Engineering
is the data science and business analytics program. I was the 
program manager for this MSc for the year 
2021 to 2022 to cover for a staff member
on maternity leave. This year the MSc had 50 students and one
of the challenges was finding supervisors for the project module.
Typically I used to supervise 2 MSc projects in data science per year (and do
4 vivas), but given the growth in student numbers,
I expect to supervise around 6 MSc projects per year in the future.

Titles of completed MSc projects
\begin{description}

\item[2017-2018] The relationship between the life satisfaction of the
citizen in European countries and their academic level.

\item[2018-2019] Using machine learning models to classify and detect
  avian wildlife around wind farms using time-lapse images.

\item[2018-2019] The use of machine learning to study the performance
  of wind farms.

\item[2019-2020] Machine learning applied to Optometry/Cataracts.

\item[2019-2020] A comparison of machine learning methods for the
  prediction of locational crime rates.

\item[2020-2021] Using Convolutional Neural Networks
for Tumour Identification in
Magnetic Resonance Imaging
Brain Scans.


\item[2020-2021] Using machine learning to study the behavior of
  customers of British Telecom (BT). 

\item[2021-2022] Investigation of churn of customer of telecom
  industry using machine learning.


\item[2021-2022] Investigating customer churn in british telecom.

\item[2021-2022] Real-time data visualisation of pledge activity and trends.

\item[2021-2022] Topic Modeling With Latent Dirichlet Allocation on  mining journal to find out useful and meaningful insights.

\item[2021-2022] Incorporating machine learning into business
  intelligence using spinnaker international ltd as a case study.


\end{description}

I have co-supervised the following MSc projects.

\begin{itemize}

\item  Graphene as an attractive nanomaterial for biosensing.

\item The use of tools to collect images from social media
platforms for data collection and analysis of cultural
ecosystem services across Europe.

\end{itemize}



At the University of Wuppertal, for one year, I was a tutor in
a computer lab for a course on computational 
science, as part of a masters program.


\subsection{My experience in supervising PhD students}

I have supervised one PhD student in lattice field theory to
completion. I am co-supervising a second year PhD student in data science on
a project: ``machine learning methods for next generation customer
service'' in cooperation with BT research. I am the lead supervisor
for a first year PhD student in quantum computing. I co-supervise
a PhD student in Optometry.


I was the external examiner for PhD students
at Trinity College Dublin in 2006 and 2013,
at Cambridge University in 2018,
University of Adelaide in 2019, 
and the Universty of Swansea in 2022.


I have always worked closely with the graduate students in lattice
gauge theory at all the Universities I have worked at.  For example I
co-authored papers with 4 graduate students at Liverpool. I wrote a
paper with a graduate student at Edinburgh.  For two years I organised
a two day training workshop for postdocs and graduate students, known
as "HackLatt", on lattice QCD codes in Edinburgh. I was the originator
of the HackLatt workshops, that are now part of the culture of lattice
QCD in the UK, and have been favourably reviewed by various
international review panels.



I organized the seminars in the theoretical particle physics division
at the University of Liverpool for about five years. At Glasgow I
organized the seminars for the theory group.  At the University of
Utah, I organized an internal seminar series for particle theory,
because there were so few external speakers.
I was an active
participant in the weekly particle theory postgraduate discussion
group at the University of Liverpool.
I attend a seminar for undergraduate students at the University
of Wuppertal.
I was a tutor at the BUSSTEPP summer school for first year
graduate students in theoretical particle physics for two years.



\subsection{Teaching qualifications and teaching CPD}

I am a fellow of the Higher Education Authority, which is the standard
teaching qualification in the UK.

During my time at University of Plymouth, I have attended external
workshops on online assessment and Physics Education Research.  In
June 2022, I co-organized a one day workshop: 
"Involving employers in the development of the mathematical sciences
curriculum," supported by the RSS and IMA (\url{https://sites.google.com/view/employersandmathscurriculum/home}).

 In
September 2005, I attended a two day workshop in Birmingham about
teaching mathematics. In January 1999, I attended a very good lecture
at Liverpool about the art of lecturing.


\section{Teaching philosophy} 

\subsection{Undergraduate teaching}

I aim to train students in skills they
can either use in academic work or in
industry. I would also like the students to
learn about the many interesting 
topics in mathematics and physics, that they may never 
need professionally (for example the physics of black holes)
after they end their undergraduate
studies, but are part of our culture and fun
to follow in the popular press.
I realize that some students may only want
to pass the final exam with a reasonable mark,
and I am happy to help them do this.

I have tried various methods of providing brief
summaries of the material at the start of the lecture
and on-line. After marking too many exam scripts from
students who did not know that $i^2 = -1$, I provided
a checklist guide for revision (approved by two senior
Professors) for an exam, 
in addition to the exam papers and solutions of previous exams.

To try to more effectively teach calculus to first year students I
looked in books such as "how to ace calculus" and Polya's "how to
solve it" for hints. I also created a mind map for the different ways
of solving integrals. I only want to innovate in a small section of
the course, so that I could then measure the success, but without the risk
of a poor average mark in the final exam.

\subsection{Small group tutorials and projects}


I work very hard on getting the students to join in the discussions
and try problems in small tutorial groups. I try to create
a friendly and nonjudgmental environment (even when
students forget how to solve a quadratic equation).
I try to provide useful feedback on marked homework,
as soon as possible to the students. I learned a lot
about the difficulties students have learning mathematics
and physics, when I talked to them in small tutorial
groups.

I view essay and student projects as crucial to developing
problem solving skills. I try to design projects
that are simultaneously well structured, yet require
the students some initiative.

I usually try to create projects
that are a little bit "cool," but still doable by undergraduates.
For example in the project I designed for students
in a Masters course at Wuppertal, I got them to solve
the one dimensional Schr\"{o}dinger's equation. Towards the end
of the project I suggested they used a screened potential
to look at the melting of mesons in the quark gluon
plasma. I also hope that the additional motivation for the
project will help them,
if they interview for graduate studies.

\subsection{Graduate teaching}

My aim in graduate education is develop in the students:
independent problem solving skills, 
the ability to critically review material, and 
the ability to develop new areas of research. My 
involvement in the postgraduate discussion group at Liverpool
was aimed at developing the last two skills.

I want the students to use modern computer
techniques, such as scripting, C++, python, standard numerical
analysis libraries, rather than using some old legacy
FORTRAN program, because this will help them 
get jobs in industry, if they so wish.

%% \input grant


\section{Service to the theory community}

I have refereed papers for the journals: Phys.Rev.D, Phys.Rev.Lett.,
The European Physical Journal C, SIAM J. Matrix Anal. Appl.,
Phys. Lett. B, Journal of American Physics, Nucl.Phys. A, 
Canadian Journal of Physics,
and 
Computer Physics Communications.
I was on the local
organizing committee for the international lattice 2005 conference
held in Dublin Ireland. I was one of the three editors of the
proceedings for the lattice 2005 conference that was published in the
Proceedings of Science journal (PoS). I was on the local organizing
committee for the international conference Extreme QCD, which was held
in Plymouth 2015. I published a summary of the conference in the CERN 
courier magazine.

In 2008 I organized a collaboration meeting at Glasgow for the
European Twisted Mass lattice collaboration with 17 participants: from
France, Germany, Italy, and the Netherlands.

I organized a two day workshop at Glasgow in April 2008 with the
title: "The nuclear physics challenge to lattice QCD"
(http://nuclear.gla.ac.uk/nuclat/). The aim of the meeting was to
open a dialog between the nuclear experimental and lattice
QCD communities. There were 65 participants at the meeting.

In January 2009, I was an invited reviewer for the midterm review of
the project: "A national computational infrastructure for Lattice
Quantum Chromodynamics" funded by the US Department of Energy 
under the SciDac-II program (Scientific Discovery 
through Advanced Computing).
I attended a two day meeting in Washington DC and wrote a report
on the status of the project.

I am a member of the Institute of Physics and I can use
CPhys MInstP after my name. I am a member of the British Computer
Society.

\section{Outreach and public understanding of science}

In 2017, 2021, 2022, and 2023 I co-organized a particle physics master 
class at Plymouth, for students at local schools  (http://math-sciences.org/masterclass .)
I have run shows on dark matter, to visiting school students,
in the planetarium on campus. I have hosted small groups
of school students on work experience and Nuffield 
schemes. They worked on particle physics or quantum computing
projects.

I organized a one day festival of Physics with the South West branch of
the IOP at the University of Plymouth in 2019 and 2022. In June 2022,
I organized a visit of physics school teachers to visit
the Engineering and Physics facilities at the University of
Plymouth (\url{https://sites.google.com/view/visit-of-physics-teachers/home}).

%%\input compute


\section{Review talks at conferences}

I regularly present review talks at international
conferences with audiences of both theorists and experimentalists.
A list of my main conference presentations follows.

\begin{itemize}

\item I presented a topical plenary
      talk at the international lattice 2013 conference
      in Mainz.

\item In April 2013, I presented results of my lattice
      QCD calculations with the charm quark in two talks, at 
the 9th International Workshop on Heavy Quarkonium 2013 
at IHEP, Beijing, China.

\item In March 2013, I presented a 
review talk~\cite{McNeile:2013rga} on 
``Determination of the strong coupling and quark 
masses from Lattice QCD'' at a workshop
at the Institute for Advanced Studies, Nanyang Technological
University, Singapore.


\item I presented a talk at the 
8th International Workshop on Heavy Quarkonium 2011 
GSI, Germany between 4 - 7 October 2011. At the same meeting I
was also a panel member in round table discussion about the
importance of including charm in the sea of lattice QCD calculations.

\item I presented an invited talk 
   ``Physics from staggered fermions with light quark masses''
at the meeting:
``Hadrons, Lattice QCD and Chiral Perturbation Theory'',
September 13 - 16, 2010 Graz, Austria.

\item In November 2009 I presented an invited talk
      on resonances and lattice QCD at a
      workshop on Hadronic excitations at TRIUMF lab in Canada.

\item In August 2009 I presented an invited review talk
      ``strong decays on the lattice'',
       at the Charmed Exotics meeting in Bonn Germany.


\item In June 2009, I was invited to present a talk at 
      the ``QCD Bound States'' Workshop at
the Physics Division of Argonne National Laboratory.


\item In July 2008, I reviewed results 
      for glueballs from lattice QCD at the QCD08 conference
      in Montpellier, France.

\item In May 2008 I gave a review on
hadron spectroscopy at the international conference 
``Nuclear Structure at the Extremes'' held in Paisley Scotland.

\item In March
2008 I presented an invited review talk on lattice QCD at the
Photon-hadron physics with the GlueX detector at Jefferson Lab
workshop, in Virgina USA. 

\item In 2007 I presented a plenary review talk on hadron spectroscopy 
      at the international lattice 2007 conference held in
      Germany. 

\item  In
2007, I reviewed the spectroscopy of scalar mesons obtained from
lattice QCD at the International Symposium on Meson-Nucleon Physics
and the Structure of Hadrons (MENU2007) in Juelich, Germany.  

\item I gave a short talk at the Workshop on
Light-Cone Distribution Amplitudes at IPPP Durham (September 2006).

\item I presented an invited review talk at the ICHEP'06 conference
in Moscow (July 2006) about lattice QCD calculations of the 
new heavy hadrons.

\item I presented an
     invited talk at the meeting "Highly Excited Hadrons" in Trento (July
      2005). 

\item In September 2004, I was
invited to give a review talk in Japan at a meeting called ``Lattice
QCD simulations via International Research Network''.  

\item I reviewed the status of hybrids
and glueballs calculations from lattice QCD at the future physics at
Compass meeting at CERN in September 2002.  

\item I gave a mini-review on unquenching the $f_B$
decay constant at the UK Phenomenology Workshop on Heavy Flavour and
CP Violation at Durham (September 2000). 

\item I
reviewed the lattice QCD results for hybrid mesons and glueballs at
the European workshop on the QCD structure of the nucleon (Ferrara,
Italy) in April 2002.  

\item In March 1999, I reviewed the lattice results for hybrid mesons at the
international conference on hadron spectroscopy in Frascati.  

\end{itemize}

I have been a lecturer at the following summer schools:

\begin{itemize}

\item In July 2012, I presented two lectures
      about hadron physics and the PANDA experiment, as part of
      Helmholtz Graduate School for Hadron and 
      Ion Research (HGS-HIRe) for FAIR in Germany.

\item I presented two lectures on
heavy quarks in lattice QCD at the international school on heavy quark
physics in Dubna, Russia (June 2002). 

\end{itemize}


I have made many presentations at the annual international conference
on lattice field theory. 

Below are some recent seminars that I have presented.

\begin{itemize}

\item In April 2013 I presented the talk:
``Implications of heavy glueball results from lattice
QCD for the PANDA experiment,'' at a SFB meeting
in Regensburg.

\item In March 2011, I presented a talk:
``Status of the Regensburg-Wuppertal charm project''
at a SFB meeting in Regensburg.

\item In October 2009 I presented a seminar with the title:
       ``Decay constants in charmonium'', at the
       Helmholtz-Institut für Strahlen- und Kernphysik
       at the University of Bonn.

\item In May 2009, I presented the talk: 
      ``Using a compute grid'' at the e-science institute
       in Edinburgh.

\item I presented the talk "Lattice QCD and charm physics"  to the
      Bristol HEP group at their away day. (December 2008)

\item I presented 
      the talk: "A lattice QCD calculation of the decay 
      constants of heavy and light mesons", 
       at the University of Regensburg, Germany
      (November 2008).

\item I presented the talk: "Some decay constants from twisted mass QCD."
      at the Humboldt University in Berlin Germany (June 2008).

\end{itemize}

\section{Research plans}

I have been doing research into theoretical particle physics
since 1989. Since I have been working at the University of Plymouth,
I have started a program of research into data science.

\input cv_research_datascience

\input cv_research

\section{Other relevant skills}

In March 2012 I passed an exam in the German language
at the B1 level in the EU classification. 
This corresponds to around 300 hours of study.

\section{Publications}

\begin{figure}
\begin{center}
\includegraphics[scale=0.4,angle=0]{figures/Pubs_dec_2022.png}
\end{center}
\caption {
Summary of my publications from the Inspire database (Dec 2022.)
}
\end{figure}


\input pub_journal


\input pub_conf


\end{document}






% LocalWords:  Regensburg SFB Institut
