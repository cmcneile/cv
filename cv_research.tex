%%
%% $Id: cv_research.tex,v 1.7 2012/11/23 08:48:29 cmcneile Exp cmcneile $
%%


\section{Description of research}

Lattice QCD calculations are a crucial part of the physics programs of
both high and intermediate energy experiments at facilities such as:
KEK, Fermilab, Jefferson Lab, BES, GSI, and the LHC.  
The aim of many high energy experiments, such as the
LHCb experiment and the soon 
to start superB factory, is to 
study the properties of quarks to look for
evidence for physics beyond the standard model. However, the actual
experiments only detect hadrons, so QCD must be ``solved'' to extract
the properties of quarks. QCD is a strongly 
coupled quantum field theory that has a number of potential novel
bound states. The experimental search for these new bound states
requires accurate solutions of the theory.


The best way to compute non-perturbative
quantities from QCD is to use lattice QCD.
My research in lattice QCD is very strongly focused on phenomenology.
I have worked in two general areas: the decay properties of mesons
containing heavy quarks and lattice calculations of the properties of
glueball and hybrid mesons. 

\subsection{Overview of research goals}

My long term research goals are:
%%\begin{itemize}{labelitemi}{$\star$}

\begin{itemize}
\item Discovering the effects of physics beyond the standard model
      using the results of precision lattice QCD calculations
      and experimental data.

\item Using lattice QCD to map out the mass spectrum
      of charmed baryons.

\item Using lattice QCD to predict the mass spectrum of 
      exotic charmonium mesons.

\item Using lattice QCD to identify glueball degrees of freedom
      in hadron spectroscopy.

\end{itemize}


My long term research goals are important because
of the following reasons.
\begin{itemize}

\item All my research is required for major
      experiments in particle and nuclear physics.

\item At this moment, the LHC has not provided direct
      evidence for physics beyond the standard model
      of particle physics, so indirect tests based
      on flavour physics are crucial to determining the
      energy scale that new physics exists at. If the LHC
      does directly discover new particles, flavour
      physics observables will help map out the theory.

\item There was a lot of press coverage when the $\eta_b$ meson
      was recently discovered, so I would expect that the experimental 
      detection of exotic mesons, double charmed baryons, or
      glueballs would also generate a lot of coverage in the
      media.

\end{itemize}

Ultimately I would like to understand the structure of the 
CKM matrix and the values of the 
quark masses~\cite{McNeile:2010xe}.


%%
%% B physics
%%

\subsection{Flavour physics and lattice QCD}

I am an expert in the research field: "lattice flavour physics".  The
aim of this research is to determine CKM matrix elements and quark
masses very precisely, and thus to hopefully find evidence for, or
constraints on, physics beyond the standard model of particle physics.
These calculations are crucial to the success of particle physics 
experiments at the intensity frontier.

\subsubsection{Research highlights in Particle Physics}



Recently I have been working with the HPQCD collaboration on
precision charm and bottom physics. 
These projects have produced the most precise
value of the QCD strong coupling from 
any method~\cite{Davies:2008sw,McNeile:2010ji}. 
The mass of the
charm quark was computed with an accuracy of order 1\%, using
a novel method based om computing moments of the pseudo-scalar
correlator~\cite{McNeile:2010ji}. 
By computing the ratio of the charm quark to strange quark 
mass and exploiting the
accurate calculation of the charm quark, we obtained a value
for the strange quark mass accurate at 
the 2\% level~\cite{Davies:2009ih}.


With the HPQCD collaboration I have computed 
the $f_{D_s}$ decay
constant~\cite{arXiv:1008.4018}
at smaller lattice spacings
than HPQCD's original published result.
The original HPQCD calculation of $f_{D_s}$
disagreed with the experimental determination
by CLEO-c, BaBar and Belle by $3.8\sigma$. 
The new calculation of the $f_{D_s}$ decay
constant is now 1.6$\sigma$ lower than
the most recent experimental determinations.
(The experimental numbers have changed more than the
lattice numbers.)
The theoretical calculation of $f_{D_s}$ can
be used
with experimental results to put constraints 
on the properties of charged Higgs bosons
(which the LHC may be able to directly discover).


Lattice calculations with heavy quarks are hard because
the systematic errors are of the order heavy quark mass times
the lattice spacing to some power. The precision results obtained
in the charm sector were obtained because of the use
of the HISQ action, that has a reduced dependence on
lattice artifacts. The existence of very fine lattices 
generated by the MILC collaboration and the good properties of
the HISQ lattice action motivated us to try to
study the bottom quark on the lattice.
We found that using the HISQ action with fine lattices,
we could accurately compute: 
$f_{B_s}$ decay constant~\cite{McNeile:2011ng}, 
the mass and decay
constant of the $B_c$ meson, and the mass of the bottom quark
using the moments method~\cite{McNeile:2010ji}. 


Lattice QCD calculations provide input to other 
non-perturbative calculations, such as light cone
sum results. I have computed the transverse
decay constant of the $b_1$ and $J/\psi$ mesons
for such applications~\cite{Jansen:2009hr,Jansen:2009yh,Dimopoulos:2008ee}. 
This is part of a program
to use lattice QCD to compute a wider set of 
observables than the small set of weak matrix elements
traditionally computed in lattice QCD calculations.

In the past I have worked on a 
number of 
calculations~\cite{Bernard:2002pc,McNeile:2004wn} of
the $f_{B}$ and $f_{B_s}$ decay constants, 
because this is important for unitarity
tests of the CKM matrix. The measurement of the leptonic decay 
of the B meson is one of the key measurements of the superB factories.
Recently with the HPQCD collaboration I have computed
the $f_{B_s}$ decay constant~\cite{McNeile:2011ng}, 
which is the key non-perturbative QCD 
input to the decay $B_s \rightarrow \mu \mu$ decay.
There was a lot of press coverage 
(including the 
BBC~\footnote{\url{http://www.bbc.co.uk/news/science-environment-20300100}},
)
when the LHCb experiment recently claimed a ``measurement''
of this decay.


%%
%%  BMW staggered 
%%
\subsubsection{Future research plans in Particle Physics}

The BMW-c collaboration has generated a large number
of unquenched gauge configurations, using the 2 HEX clover quark action,
with a range of light pion masses 
that go down to below the physical pion mass.
I have been using these configurations as the basis
of lattice QCD calculations that include the charm quark.
This work is done in collaboration with a group at the
University of Regensburg under a SFB grant, so I
refer to those projects as SFB.

I have nearly finished a calculation of the mass of the 
charm quark. As part of this calculation an 
estimate of the $f_{Ds}$ decay will be produced.
This will be followed
by a lattice QCD calculation of 
the semi-leptonic decays of the D and $D_s$ mesons.

I am now particulary interested in calculations of quantities
where there is some tension between the standard model
predictions and experiment. For example, I have started
a lattice QCD calculation of the leading order hadronic corrections
to g-2, because this quantity is showing a deviation
between experiment and theory, and so
is an important constraint 
on the energy scale of BSM physics. Althogh it is will be very challenging
to obtain  errors below 1\%. Ultimately,
this is also a warm up exercise for the calculating the even more 
challenging light by light diagrams. I am the PI for
the computer time on this project.


The Soft Collinear Effective Theory (SCET)
approach to studying B meson decays depends 
on the inverse moment of the light cone wavefunction
of the B meson ($\Lambda_B$), as well as other 
parameters. Although lattice
QCD has been used to compute the moments of the light cone
wave functions of light mesons, there is no formalisim
to compute inverse moments. I am thinking about the feasiblity
of estimating ($\Lambda_B$) using lattice QCD, because
SCET can be used for non-leptonic decays.
There is a wide range of values for $\Lambda_B$.
For example one analysis of the semileptonic decays
of the B to $\pi$ meson
(hep-ph/0504091) finds 
$$
\Lambda_B = 460 \pm 160 \; \mbox{MeV}
$$
but I have seen other values of $\Lambda_B$ outside the
above range in the literature.

A pragmatic way to proceed is to fit the branching ratio
for a decay such $B \rightarrow \nu_l l \gamma$ 
computed using lattice QCD
to the SCET
prediction (1110.3228). 
The decay $B \rightarrow \nu_l l \gamma$ may be measured at the
super B factories and could also be important for the experimental
determination of the leptonic $B \rightarrow l \nu_l$ decay
(0907.1845).



\subsection{Hadronic physics and lattice QCD}


The experimental detection of a hybrid meson or glueball degrees of
freedom is the "holy grail" of hadron 
spectroscopy.  A crucial component
in the quest for evidence for effects of "dynamic glue"
 are robust non-perturbative calculations that solve QCD.  
I am an international expert on lattice QCD calculations
of hadron spectroscopy and I am regularly asked to review the field.


\subsubsection{Research highlights in Nuclear Physics}

I worked on
one of the first calculations of the mass spectrum 
of light and heavy exotic hybrid mesons,
while I was in the MILC collaboration~\cite{Bernard:1997ib}. 
At Liverpool I worked on the decay of the 
hybrid potential~\cite{McNeile:2002az} and light
exotic meson~\cite{McNeile:2006bz}.

With Chris Michael I have developed and applied a technique to compute
partial decay widths of mesons using lattice QCD.  We have computed 
partial decay widths for the $\rho$ meson, static exotic meson, light
$1^{-+}$ hybrid meson, $b_1$ meson and the flavour non-singlet $0^{++}$
meson.


At Liverpool, I worked on the
mixing of glue and fermionic $0^{++}$ scalar operators using
unquenched lattice QCD~\cite{McNeile:2001xx,Hart:2006ps}, 
that is important to finding the contribution
of glueball degrees of freedom to flavour singlet $0^{++}$ $f_0$ mesons.
I recently worked on an unquenched lattice QCD calculation
of the glueball spectrum~\cite{Richards:2010ck,Gregory:2012hu}.


I have worked on the spectroscopy of heavy-light mesons,
such as determining the mass of the 
$D_s(2317)$ state~\cite{Dougall:2003hv},
that had experimental properties that were unexpected.
I worked on the first lattice QCD calculation
of the OZI suppressed
contributions to charmonium 
mass spectroscopy~\cite{McNeile:2004wu}.


I have worked on a number of lattice QCD calculations 
of the masses of the $\eta$
and $\eta'$ mesons from lattice QCD with 
2~\cite{McNeile:2000hf} and 2+1~\cite{Gregory:2011sg} flavors of sea
quarks. The flavour singlet pseudo-scalar mesons are harder to study
on the lattice because the correlators are more noisy, so much higher
statistics are required than for studies of flavour non-singlet
mesons. I have been using the improved staggered fermion formalism to
do these lattice calculations. The computation of the mass of the
$\eta'$ meson is also regarded as a crucial theoretical test of the
improved staggered formalism, that has been widely used in
phenomenological studies. These papers also included a study
of $\eta-\eta^\prime$ mixing, which is important to understand 
decays involving the $\eta$ or $\eta^\prime$ meson.

\subsubsection{Future research plans in Nuclear Physics}


As part of the SFB project with charmed quarks I am computing the spectroscopy
of charmed baryons. The widths of the charmed baryons are much smaller
than those of light baryons, so this should make it easier to
do precision lattice QCD calculations without using specialized
techniques for resonances. There are a number of charmed baryons that 
have not been discovered experimentally, such as the double and 
triple charmed baryons, so lattice results are predictions.
I am also looking at the parity
partners of the charmed baryons, which have never been studied
before using lattice QCD techniques.
Some preliminary results for charmed baryons were presented at the 
Quarkonium working group meeting
in 2011. I would like to map out the spectrum of 
charmed baryons as this could give insight to the more
complicated light baryons.


An important area for my research in hadron spectroscopy,
is to compute the spectroscopy of charmed exotic
mesons. The experimental detection of these states will be
one of the main highlights of the PANDA experiment at FAIR,
that is planned to start just after 2018. There are some
preliminary results from the SFB project~\cite{Bali:2011dc},
but I want to do the calculation with the 2 HEX clover configurations.


As part of our recent work on 
unquenched glueballs~\cite{Gregory:2012hu}
I realized that
there is a disagreement in the literature for masses of glueballs
with exotic $J^{PC}$ quantum numbers from different groups.
The problem is to do with the way continuum $J$ quantum number
is assigned from lattice representations. In the continuum limit
one group sees two glueballs with exotic $J^{PC}$, but another 
claims to see none. One of the main goals of the PANDA experiment
at GSI is search for exotic glueballs, so this issue must be clarified
with an additional calculation.

I would like to update the analysis of $\eta-\eta^\prime$ mixing
to include the pseudo-scalar glueball. This is an unresolved issue,
and is of interest to experimentalists working on the KLOE
experiment in Italy and others.


