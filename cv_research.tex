\subsection{Research plans in theoretical particle physics}
%%
%%  g-2
%%

The exciting experimental measurement of the anomalous magnetic
moment of the muon ($a_\mu$) from the Fermilab Muon g-2  experiment,
combined with the
earlier result from BNL, raises the hope of a measurable failure of the
standard model of particle physics.  However, a recent lattice QCD
calculation of the leading hadronic order hadronic correction to
$a_\mu$
(\amu)
from the BMW collaboration found better agree with experiment,
thus motivating further improvements in the lattice QCD
calculations. I have been working with the FNAL lattice, HPQCD, MILC
collaboration to compute \amu.

I am leading the effort in the FNAL lattice,HPQCD,  MILC
collaborations to compute the required disconnected diagrams and
QCD+QED contributions to \amu. This calculation, combined with
others, is a crucial component of the effort to reduce the error on \amu
to under 0.5\%.
The results on disconnected QCD contributions and connected
QCD+quenched QED are being finalized with results at three lattice spacings.

After the previous work has been finished, I will work on
computing the disconnected QCD+QED contributions and the contribution
of QED and QCD in the sea. I have been working on getting a HMC
code, originally developed by the MILC collaboration, that includes
the dynamics of QCD+QED in the sea, ready for production running.
Other groups have found these contributions to be small, so the
calculations will be challenging. 


%%

%%
%%  spectroscopy
%%

Understanding the experimental X,Y, and Z mesons in terms of quark and
glue degrees of freedom is still a hot topic. With a PhD student,
I have computed the mass of the $1^{-+}$ hybrid ($\overline{q}q$
mesons with excited glue) meson in charmonium in the continuum limit
for the first time. This extends the work on the precision study of
the properties of charmonium by the HPQCD collaboration.
The next step will be to include scattering states in the calculation.
Another direction planned is to look at hybrids with bottom quarks,
which is possible with the HISQ formulation 

I will investigate a variety of machine learning algorithms
focused on reducing the statistical errors on phenomenologically
relevant quantities computed in lattice QCD calculations. For example,
I will study symbolic regression, which can extract a fit model from
data without human input, by cleverly searching the huge parameter
space. I am currently investigating the use of Gaussian processes
for continuum extrapolations.

The staggered fermion formalism is the only formalism that
doesn't use multi-grid matrix inverters. A formalism has been developed
and implemented in the open source QUDA library, but the algorithm has
many parameters to tune. I will investigate the use of the
hyper-parameter search techniques commonly used in deep learning to tune
parameters of the multi-grid algorithm.

I am supervising a PhD student who is studying the Variational
Quantum Eigensolver (VQE) on quantum computers, with application
to lattice QCD applications.


